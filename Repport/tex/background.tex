
\section{Background}

This section will touch upon the subjects of the varying popularity between
 board games and video games, and try to explain some of the main reasons why
 they are varying.\newline
\\
The concept of playing games has existed for a very long time, but its meaning
 has changed through time. Just a decade or two ago computers were not as
 normal to possess as they are now, and by “playing games” people usually
 referred to some game involving physical activity or playing with physical
  objects. Chess is a good example of this and it is probably one of the
  oldest games that are still being played today.
Families would usually gather around the table at late evenings and have a
good time together before bedtime.\newline
\\
Due to technological advancement, computers now have a fixed spot in the
 homes of almost any family living in modern societies, and the term “games”
 has reached a much broader meaning. Whereas people who grew up in the
  60 - 70’s would most likely “play games” in shape of physical board
  games like chess, play cards or something similar, today’s generation
  has a much wider range of opportunities, such as computers and gaming
   consoles. And it seems to be the trend to play games through computers
    and consoles, because as an article from telegraph.co.uk states it with
     a survey:
\begin{displayquote}
"The survey suggested that while 73 per cent of parents remembered regularly
 playing board games as children, only 44 per cent of the children polled said
 they do so now"\footnote{http://www.telegraph.co.uk/men/relationships/fatherhood/11696224/Card-games-and-board-games-are-dying-out-and-its-no-great-loss.html}
\end{displayquote}
The reason for the board games' gradually decreased popularity, besides
 the technological progress, may also be connected with the games themselves.
  As board games are games involving physical objects, their value of
   experience is thereby limited. In order to play board games the participants
   have to be physically present at the same place as the games take place in
   the physical world. Video games, on the other hand, takes place in the
   digital realm and thus they offer another level of experience. But the
   thing that makes video games truly special, and has been a core factor
   of video games’ success in recent years is the concept of online
    multiplayer.\newline
\\
An online multiplayer means you have the opportunity to play against other
players from all around the world. This means that when people play games
online on a multiplayer game, they very often play with or against people
they have never met or seen before. This concept is often called
“the separation of space and time” and it is one of the major by-products
 of technological advancement; you basically chat and interact with players,
  even if they are sitting on the other side of the planet, and the responses
   are instant.
Since board games are physical, they are subject to break sooner or later or
at least miss a piece of the game thus making the game impossible to play.
 Video games does not have this problem since they have no physical shape.
 All they take to play is an installation and they do not age the same
 way. \newline
\\
If board games, or just the games, are to survive in a time with video games
gaining popularity, the games must be digitized so it will be possible to play
 on the devices most people use these days: The computer, console or
 smartphone. \newline
\\
Ultimately this means it is no longer a board game when it has been digitized.
 But the game is the same - it has the same rules. But the way it is played is
  just different.\newline
