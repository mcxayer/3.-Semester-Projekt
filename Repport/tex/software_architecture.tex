
\chapter{Software Architecture}
In this section the group will describe how it used software architecture
 in the project, the different choices the group made to ensure that the
  system runs according to the specifications, defined by the group, and
   how the work was planned, executed and then possibly changed so it would
    work better. It will also describe the structure of the system and how
    the system has been distributed.
\\
\section{Architecture}
As told earlier in the report the group has chosen to make an online version
of the game Battleship. With this in mind the group generated several
requirements in the form of use cases, these use cases are listed in the
Requirements chapter and in that section it is also shown how they were
prioritized by the group. The requirements were split into groups based
on what part of the system they dealt with, and prioritized using MoSCoW
as explained in the section.
\\
\subsection{Client-Server}
The main goal of the project was to make a distributed system. This the group
accomplished by separating the system into three distinct components, the
client, the server and the database. These three components run on different
 computers, the client and server are two standalone applications.
\\
\\
A distributed system is a collection of independent computers. Each computer
has its own processor and memory. This allows each computer to work
independently of each other. In order to utilise the resources of each
computer, the computers are connected via various communication architectures.
 Examples of such a communication architecture include buses and wireless
  connections.\footnote{Silberschatz et al. 2010, p. 719}
\\
\\
One way to set up the computers is through a client-server architecture. A
computer acts as a server, whilst other computers act as clients or other
servers, such as a database server. The benefits of this setup are resource
sharing, computation partitioning, connection-reliability and communication
capabilities.\footnote{Silberschatz et al. 2010, pp. 720-721}
\\
\\
Since the group decided to make a game, a client-server pattern was the
obvious choice, especially when the system had to be distributed, the client
was separated so it could run as a stand-alone application. The pattern also
makes the server accessible from different places at once, and allows the
player to play against each other online.
\\
\\
In relation to games, distributed systems allows multiple player clients to
connect to a centralised server, working as an interface for all client
requests. Examples of client requests include verifying account information,
updating the game state, or communicating with other player clients.
\\
\\
One of the major problems with the client-server architecture is the strain
on the server bandwidth. In MMORPGs such as World of Warcraft or EVE Online
thousands of concurrent player clients connect to a server. This is possible
by utilising multiple servers in their own distributed system. This allows
the servers to distribute the computation load to multiple
servers.\footnote{Winn et al. 2013, pp. 3-4}
\\
\subsection{Layer}
The group chose a layered approach to ensure maintainability and to make it
so that the game easily could be separated into its components. The group
chose to make it a 3-layer architecture pattern where the game was split
into 3 parts: (1) Presentation layer, (2) Business layer and (3) Data layer.
\\
\\
\textbf{Presentation layer:} : In this layer the interface is located.
It is here that the group developed the GUI and made use of facades.
This was to ensure maintainability since it means that the GUI or any
other part is relatively easy to change out, as there is no direct
connection between the layers. As long as the facades remain the same
the system does not care how the GUI looks or how it works.
\\
\\
\textbf{Business layer:} Here is where all the logic of the game is
located. This is where the game functionality lies and how it does it
is defined. How the connection between the server and the clients worked
 will be explained in the Service-Oriented sub section.
\\
\\
\textbf{Data layer:} This is where the database is located, again
accessible by a facade to make maintainability easy and separate concerns.
 This is a separate component in this system as the database is is not in
 the same place as the server or any of the clients. The database is where
  all the reusable information is stored. Information like user-information,
  usernames, e-mails, passwords etc. is stored in the database. The data is
   sent over a secure connection and the information is encrypted to make sure
    that none of the information given by the users can be stolen.
\\
\subsection{Service-Oriented}
The server follows a service-oriented architecture (SOA), allowing client
 and server to communicate through a common interface.
 The service-orientation works by making interfaces that act as facades
  for logic. As explained earlier this is to ensure that the logic can
   be maintained easily on the server, and easily on the client by developers
    without server knowledge.
\\
\\
The advantages of SOA is distributing a reusable, loosely coupled and
 standardized service, allowing anonymous access from clients, where
  messages can be either synchronous or asynchronous. It is defined by
  a contract and a concrete implementation, where the client uses the
  contract to access the
  implementation.\footnote{Vogel et al. 2011. pp. 206-207}
\\
\subsection{Security}
In order for a user to be authorized and authenticated in the game,
they first need to login. This is done using a single sign-on (SSO)
architecture, where a token is created and provided to the client upon
successful login.\footnote{Vogel et al. 2011. p. 220} The server tracks
the session of the token and controls how it is used. The client now does
not need to login again, but simply use the provided token. In this project,
this token can be used to connect to the game service in order to start
 playing against others. This has the advantage of allowing a client to
  use any service that supports the token after logging in. The disadvantage,
   though, is that all services supporting the token need a connection to
   the service verifying the token.
\\
\section{View Model}
There are several different architectural view models, but the group decided
to work with the 4+1 architectural view model, as this model revolves around
 the use of use cases, which is a main point of the group’s Scrum/Kanban
 development approach. The model contains six views: (1) Use case view, (2)
  Logical view, (3) Implementation view, (4) Data view, (5) Process view,
  (6) Deployment view. Looking through the project, the group has primarily
   been working with the Logical and Implementation views but have used most
   of them to various degrees. Most of the project have been spent on the
    implementation of the system and during that time, the group mostly look
    at it from the Implementation view. During the planning of the project
     it was mostly looked at from the Logical view, though many of the
     requirements were defined as use cases it is clear that the Use case
     view was used. Moreover since the system is distributed it has also
     been looked at from the Process view.
\\
\\
\textbf{Use case view:}
\\
This is the basis for the 4+1 view model, it is from this view that all the
 other views make their architectural decisions. To show all the use cases
 and how they interact with each other and the user, the group has made a use
  case diagram, this diagram has already been shown in the Requirements chapter.
\\
\\
\textbf{Logical view:}
\\
This view looks at how the use cases will be implemented. It is used to get
 an overview of the the code is going to look, how the different classes will
  work together, and what methods and attributes the classes will contain. The
   group made a class diagram for each package in the project. They can be
    seen in the appendix \ref{appendix:class_diagrams}
\\
\\
\textbf{Data view:}
\\
This view is used to describe the data models and how they interact with the
 system. Since the group used ADO.NET to make the database, an entity model
 was created and from this, the database was automatically created. The entity
  model can be seen and is explained in the chapter Implementation.
\\
\\
\textbf{Process view:}
\\
This view focuses on the behaviour of the system while it is running, and
how the different use cases cause different classes to interact with each
other and how. To show this the group developed some sequence diagrams, but
 after the first few sprints it was decided that they were not used enough
 to warrant the time spend on creating them.
\\
\\
\textbf{Deployment view:}
\\
This view describes how the static artefacts are deployed in the physical
world. This the group has shown by making a deployment diagram, which shows
how the system is separated into three parts, and how they connect to each
other. Here is the deployment diagram shown:
\begin{figure}
\centerline{\includegraphics[scale=0.65]{Deployment} }
\caption{Deployment Diagram}
\end{figure}
