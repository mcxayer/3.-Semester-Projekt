\chapter{Discussion}
	In this chapter we will briefly go over our accomplishments in the project
   period and what we think could have been done differently. There will also
    be a short description on how we used the courses of the semester in the
     project.
	\\
	\\
	We started out with the idea of making an online multiplayer turn based
   game which had the functionality of battling against each other. Through
    the semester we have expanded the project with the knowledge we learned
     from our courses.
	\\
	\section{Did we reach our personal goals?}
	We did manage to develop a simple online game where multiple people can play
   against each other. We would have liked to get more graphical in our game,
    but this was not in the requirements so we made a simple GUI instead of
    going after an advanced one, so we could get more of our use cases
     implemented.
	\\
	\\
One of the major things we wanted to learn while making this game, was how
 to use the programming language C\#, and how it was different from Java.
 We all learned to varying degrees to code in C\#. This is a great benefit
 for our future, as it is a language that is often used and therefore very
 a useful skill to have.
	\\
	\section{Did we reach the goal for our semester?}
	In terms of distribution and networking, we successfully implemented a
  distributed product, allowing connections between client and server, and
   server and database.
We tried our best to integrate the different courses into our project. Our
 plan was to implement encryption between the client and server, but due to
  time constraints we couldn’t. So now we only have a token system that
   verifies the user connecting to the server, but the password is send as
    a plain text file over the network, instead of being encrypted. We
    integrated CCM in the form of our market analysis. OPN was integrated
     as our code, and was fulfilled since we managed to create a distributed
     system. Finally DES was integrated as a guideline for our architecture
     where we split the code up into three main layers: Client, Server and
     Database. It made more sense to the group and made it so we had a good
      view of the code.
	\\
	\section{Did we follow SOLID and GRASP?}
	We did not quite follow the guidelines from SOLID and some of the patterns
   from GRASP, which we should change if we were to continue working on the
    game. The reason we lost sight of using them, might be because most of
    us had not used C\#,  WCF, or WPF before. So our focus was instead to
    try and implement a working distributed system.
	\\
	\section{Could we sell the game as it is now?}
	If we want to sell the game ourselves then no, but we would be able to
  sell the game if we fixed the things mentioned above. It is hard to say
   if we could sell the game on the international market, since our market
    analysis does not go into great detail how the international market
     would respond to such a game. If we were to go on the international
     market, it would be wise to make a new analysis that looks into how
     the market would respond and whether or not it would be viable for us
      to do so. This would imply that we would have to maintain the server
       and the cost would have to be covered by the sale of the game. However,
        as our market analysis shows if we use Steam we could reach an
         international audience, but we do not know if it would sell.
	\section{Project Problems}
	During this project there has been some issues that the group have had to
   deal with. One was that the group chose to write the code in C\#, which is
    a programming language that most of the group had not worked with before.
    The reason the group chose to write in C\# was to expand the skillset of
     the group, but it caused the development of code to be slow since the
     group had to learn the differences between Java and C\#. One of these
      differences was that the group was taught how to network in Java and
      that it would be perfect for client-server interaction. Because of that
      the group tried to find C\# equivalents, which WCF was, culminating in
       the use of this in the project.
	\\
	\\
	The distribution of the system is not currently across multiple machines but
   rather only locally. This is due to several issues with compatibility of
   certain technologies and time involved. The provided Linux server can not
    directly use .NET without some intermediary. The mono framework can be
    used to replace .NET, but we did not have time to get this installed, as
    we were still trying to figure out .NET. In terms of the database, the
     provided one used PostgresSQL, which requires additional setup in Visual
      Studio in order to get to work with the ADO.NET entity model framework.
       We did initially work with a PostgresSQL library to create a connection
       to the database and work directly with SQL queries, but it quickly
       became time-consuming. In the end we settled with keeping it locally,
        but still open for change in terms of connectivity.
	\\
	\section{Project successes}
	On the other side there were also some things that the group did very well.
   In this project the group have worked very well together without any
   problems in the group, and everyone have done their share. Also the use
    of Kanban made the development of the software very structured and easy
     to keep an overview on.
	\\
	\\
If there was anything the group should have done differently it might be to
 make sure the entire group had studied more on the new programming language.
  There were a few in the group that had worked with it before, but the rest
  had never touched it. If those that had never touched it had studied it more
   diligently then maybe there would not have been the problem of slow code
    development. Otherwise we should have written either server or client in
    java, and the other in C\#.
