
\section{Test}
	In this section all test done on the project will be described.
  Due to time constraints, we were not able to complete any real testing.
   Instead we have described how we plan to test the elements of the software.
		\subsection{Test - Code}
Normally we should have done unit testing to make sure our code is working
 and its behavior is expected, but because we learned about Test-driven
 development very late in the semester, we did not have the time to implement
 it. Instead we relied on software testing by running our code, then checking
  if the behavior is what we expected when navigating the GUI. This is not
  optimal, since with unit testing we can keep track of which parts have been
   tested. Also once a test is written it is much faster to run the test again,
    than having to run the code each time to check if the problem was fixed.
    This would have saved us a lot of resources in the projects, so if not
    because of the late introduction to unit testing, we might had saved
    some time.
\\
\\
Another way to test the code would be to crudely follow the sequence of events
 for the use cases, and verify if the code matches the expected behaviour,
  as well as pre- and post conditions.
		\subsection{Test - Server}
Since the game is a multiplayer, we need to make sure that our server can
 hold that many clients accessing the server. We would test our server’s load
 and performance by stress testing it. This is done by creating a lot of
  virtual users creating an account, logging in, searching for matches and
  makes moves in the games. Simultaneously we would create virtual users
  that log in and out, and have others that just stay in the lobby.
\\
\\
This test will simulate how the game will be used by users, and the server
 should therefore be able to handle such load, or else we need to optimize
  our server performance.
		\subsection{Test - Game}
The purpose for this test is more for the functionality and general feel for
 the game, and should be tested with real users doing user acceptance testing.
\\
\\
We could go about this in different ways:
	\begin{enumerate}
		\item Getting different focus groups to play the game and after that
    interview them.
		\item Publish the game prematurely to a fixed amount of users, then get
    as much feedback as possible. Also known as alpha/beta testing.
		\item Publish the game prematurely to all users, while still developing
    the game. Feedback will be collected from users that voluntarily chooses
    to do so. Also known as “Early access” and is a form of alpha/beta testing.
		\item A combination of test 1 and 2. First do test 1, then test 2 if the
    feedback was great.
	\end{enumerate}
Another test of the game is to test the game for bugs, which could be done
together with test case 1 and 2 stated above, or just during development since
 our game is quite small. It would still be possible with 3, but is not
 desired since the general public will play it, and the game should therefore
  be kept to a version that is stable. Test 4 could be desired if we were
   developing a major game, but since not, we would probably go with test
    1 as our test case.
